\documentclass[10pt,a4paper]{report}
\usepackage[utf8]{inputenc}
\usepackage[portuguese]{babel}
\usepackage[T1]{fontenc}
\usepackage{amsmath}
\usepackage{amsfonts}
\usepackage{graphicx}
\usepackage{lmodern}
\usepackage{amssymb}
\usepackage{verbatim}
\usepackage{float}
\usepackage{minitoc}
\usepackage{hyperref}
\usepackage{listings}
\title{\LARGE{Iptables}}
\date{}
\renewcommand{\mtctitle}{Conteúdos}

\usepackage{listings}
\usepackage{xcolor}
\definecolor{codegreen}{rgb}{0,0.6,0}
\definecolor{codegray}{rgb}{0.5,0.5,0.5}
\definecolor{codepurple}{rgb}{0.58,0,0.82}
\definecolor{backcolour}{rgb}{0.95,0.95,0.92}

\lstdefinestyle{mystyle}{
    backgroundcolor=\color{backcolour},   
    commentstyle=\color{codegreen},
    keywordstyle=\color{magenta},
    numberstyle=\tiny\color{codegray},
    stringstyle=\color{codepurple},
    basicstyle=\ttfamily\footnotesize,
    breakatwhitespace=false,         
    breaklines=true,                 
    captionpos=b,                    
    keepspaces=true,                 
    numbers=left,                    
    numbersep=5pt,                  
    showspaces=false,                
    showstringspaces=false,
    showtabs=false,                  
    tabsize=2
}

\lstset{style=mystyle}

\usepackage{titlesec}
\titleformat{\chapter}[display]
  {\normalfont\bfseries}{}{0pt}{\Huge}

\begin{document}
\chapter{Iptables}
\section*{Introdução}
Começamos por limpar todas as regras anteriores e configurar política padrão:
\begin{lstlisting}[language=Python]
sudo iptables -F
sudo iptables -P INPUT DROP
sudo iptables -P FORWARD DROP
sudo iptables -P OUTPUT ACCEPT
\end{lstlisting}
Na política padrão partimos do principio que todo o tráfego de entrada deve ser descartado e o de saída deve ser aceite.

\section*{Regras para restrições}
Definimos as seguintes regras de forma a permitir $ping$ apenas da máquina $gcc$ (10.101.151.5) e sub-rede 10.101.85.0/24:
\begin{lstlisting}[language=Java]
sudo iptables -A INPUT -p icmp --icmp-type echo-request -s 10.101.151.5 -j ACCEPT
sudo iptables -A INPUT -p icmp --icmp-type echo-request -s 10.101.85.0/24 -j ACCEPT
\end{lstlisting}
Estas regras consistem em adicionar exceções á política de entrada para pacotes $icmp$ dirigidos aos $IP$ relevantes. Para testar o seu funcionamento testamos a utilização do comando $ping$ em várias máquinas, permitidas e não permitidas, e verificamos que apenas a máquina $gcc$ e as máquinas na sub-rede 10.101.85.0/24 conseguiram efetuar o $ping$ com sucesso.\\
\\
Permitir conexões do cliente $Tintolmarket$ de qualquer origem para o servidor $TintolmarketServer$ no porto default (12345):
\begin{lstlisting}[language=Java]
sudo iptables -A INPUT -p tcp --dport 12345 -j ACCEPT
\end{lstlisting}
Tal como nas regras anteriores adicionamos uma exceção á regra de entrada, desta vez aceitando pacotes $tcp$ que tenham como destino o porto 12345. Como teste, verificamos todas as máquinas na rede e concluímos que conseguiam ligar-se ao servidor através do cliente $Tintolmarket$.\\
\\
Permitir ligações $ssh$ apenas da máquina $gcc$ e sub-rede $DC1$, $DC2$ e $DC3$ (máscara 255.255.255.224):
\begin{lstlisting}[language=Java]
sudo iptables -A INPUT -p tcp --dport 22 -s 10.101.151.5 -j ACCEPT
sudo iptables -A INPUT -p tcp --dport 22 -s 10.121.52.0/27 -j ACCEPT
sudo iptables -A INPUT -p tcp --dport 22 -s 10.101.52.0/27 -j ACCEPT
\end{lstlisting}
Adicionamos mais duas exceções na política de entrada, desta vez para ligações no porto 22 (porto default $ssh$). Para testar o seu funcionamento iniciamos o servidor $ssh$ na máquina de teste, com o comando:
\begin{lstlisting}[language=Java]
sudo /usr/sbin/sshd -D
\end{lstlisting}
E utilizamos o em várias máquinas:
\begin{lstlisting}[language=Java]
ssh -X ip-destino
\end{lstlisting} 
Garantindo que apenas as máquinas permitidas conseguiram efetuar a ligação com sucesso.

\section*{Regras para serviços utilizados}
Limitar $ping$ à frequência máxima de 3 $pings$ por segundo:
\begin{lstlisting}[language=Java]
sudo iptables -A OUTPUT -p icmp --icmp-type echo-request -m limit --limit 3/second --limit-burst 3 -d 10.0.0.0/8 -j ACCEPT
sudo iptables -A OUTPUT -p icmp --icmp-type echo-request -j DROP
sudo iptables -A INPUT -p icmp --icmp-type echo-reply -s 10.0.0.0/8 -j ACCEPT
\end{lstlisting}
O conceito destas regras consiste em aceitar tráfego $icmp$ a um limite máximo de 3 pacotes por segundo para endereços na mesma sub-rede local (10.0.0.0/8) e de seguida fazer $drop$ de todos os pacotes $icmp$ que não se enquadrem nesta regra. É também necessário aceitar as respostas, pelo que se deve adicionar também uma nova regra de $input$. O teste consistiu em fazer $ping$ a outras máquinas e, através da $flag$ $-i$ variar o intervalo entre $pings$. Verificamos que para intervalos inferiores a 0.33 segundos o envio de alguns pacotes era negado.\\
\\
Permitir $ssh$ apenas para a máquina $gcc$:
\begin{lstlisting}[language=Java]
sudo iptables -A OUTPUT -p tcp --dport 22 -d 10.101.151.5 -j ACCEPT
sudo iptables -A OUTPUT -p tcp --dport 22 -j DROP
\end{lstlisting}
Novamente, adicionamos uma regra que permite aceitar ligações no porto 22 do endereço da máquina $gcc$ e outra para descartar ligações $ssh$ em qualquer outro endereço. Para testar utilizamos novamente o servidor $ssh$ em várias máquinas, incluindo $gcc$, e cliente na máquina de teste ,variando o campo $ip-destino$, verificamos que apenas era possível efetuar ligação com a máquina $gcc$.

\section*{Regras default}

Por fim, executamos as regras mencionadas no enunciado, começando por permitir tráfego $loopback$:
\begin{lstlisting}[language=Java]
sudo iptables -A INPUT -i lo -j ACCEPT
sudo iptables -A OUTPUT -o lo -j ACCEPT
\end{lstlisting}

E permitir tráfego relacionado com uma ligação já estabelecida:
\begin{lstlisting}[language=Java]
sudo iptables -A INPUT -m state --state ESTABLISHED,RELATED -j
ACCEPT
sudo iptables -A OUTPUT -m state --state ESTABLISHED,RELATED -j
ACCEPT
\end{lstlisting}
\end{document}