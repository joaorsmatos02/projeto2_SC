\documentclass[10pt,a4paper]{report}
\usepackage[utf8]{inputenc}
\usepackage[portuguese]{babel}
\usepackage[T1]{fontenc}
\usepackage{amsmath}
\usepackage{amsfonts}
\usepackage{graphicx}
\usepackage{lmodern}
\usepackage{amssymb}
\usepackage{verbatim}
\usepackage{float}
\usepackage{minitoc}
\usepackage{hyperref}
\usepackage{listings}
\title{\LARGE{Iptables}}
\date{}
\renewcommand{\mtctitle}{Conteúdos}

\usepackage{listings}
\usepackage{xcolor}
\definecolor{codegreen}{rgb}{0,0.6,0}
\definecolor{codegray}{rgb}{0.5,0.5,0.5}
\definecolor{codepurple}{rgb}{0.58,0,0.82}
\definecolor{backcolour}{rgb}{0.95,0.95,0.92}

\lstdefinestyle{mystyle}{
    backgroundcolor=\color{backcolour},   
    commentstyle=\color{codegreen},
    keywordstyle=\color{magenta},
    numberstyle=\tiny\color{codegray},
    stringstyle=\color{codepurple},
    basicstyle=\ttfamily\footnotesize,
    breakatwhitespace=false,         
    breaklines=true,                 
    captionpos=b,                    
    keepspaces=true,                 
    numbers=left,                    
    numbersep=5pt,                  
    showspaces=false,                
    showstringspaces=false,
    showtabs=false,                  
    tabsize=2
}

\lstset{style=mystyle}

\usepackage{titlesec}
\titleformat{\chapter}[display]
  {\normalfont\bfseries}{}{0pt}{\Huge}

\begin{document}

\chapter{Snort}
\section*{Regras}
Começamos por definir os $preprocessors$:
\begin{lstlisting}[language=Python]
preprocessor frag3_global
preprocessor frag3_engine
\end{lstlisting}
E uma variável que contém o $ip$ da máquina onde corre o $TintolmarketServer$:
\begin{lstlisting}[language=Python]
var TINTOL_SERVER 10.101.204.4/8
\end{lstlisting}
A primeira regra a ser definida é a seguinte:
\begin{lstlisting}[language=Java]
alert tcp any any -> $TINTOL_SERVER 1:1023 (msg:"Foram recebidas na maquina 5 ou mais ligacoes TCP para portos inferiores a 1024 num intervalo de um minuto";threshold:type both, track by_dst, count 5, seconds 60;sid:10001)
\end{lstlisting}
Nesta regra definimos um alerta no server caso receba 5 ou mais ligacoes TCP para portos compreendidos entre 1 e 1023 num intervalo de um minuto. Para tal, usamos o $threshold$ com tipo $both$, de forma a gerar apenas um alerta no intervalo definido. O $tracking$ é feito $by\_dst$, de forma a que gerado apenas um alarme, independentemente da máquina que inicia as ligações. O $count$ foi definido a 5 e o intervalo a 60 segundos.\\
\\
A segunda regra é:
\begin{lstlisting}[language=Java]
alert tcp any any -> $TINTOL_SERVER 12345 (msg:"Possivel ataque da aplicacao NoTintol"; threshold: type both, track by_src, count 4, seconds 15; sid:10002;)
\end{lstlisting}
Aqui o alerta é definido caso ocorram mais de 3 ligações/tentativas de ligação num intervalo de 15 segundos no porto default do $TintolmarketServer$, 12345. O principio utilizado é semelhante ao utilizador, no entanto utiliza-se $tracking$ $by\_src$ para que possa ser gerado um alarme dentro do intervalo de tempo, por IP de origem, i.e., se existirem duas máquinas a correr o $NoTintol$, serão criados dois alertas dentro do mesmo intervalo de tempo. O count foi definido a 4 e o intervalo a 15 segundos.
\clearpage
Estes comandos foram incluídos num ficheiro $snort.conf$, com o seguinte conteúdo:
\begin{lstlisting}[language=Java]
preprocessor frag3_global
preprocessor frag3_engine

var TINTOL_SERVER 10.101.204.4/8

alert tcp any any -> $TINTOL_SERVER 1:1023 (msg:"Foram recebidas na maquina 5 ou mais ligacoes TCP para portos inferiores a 1024 num intervalo de um minuto";threshold:type both, track by_dst, count 5, seconds 60;sid:10001)

alert tcp any any -> $TINTOL_SERVER 12345 (msg:"Possivel ataque da aplicacao NoTintol"; threshold: type both, track by_src, count 4, seconds 15; sid:10002;)
\end{lstlisting}

\section*{Execução}
O ficheiro $snort.conf$ foi executado através do seguinte comando:
\begin{lstlisting}[language=Java]
sudo snort -c snort.conf -A console
\end{lstlisting}
Que permite visualizar os alertas diretamente na consola.

\section*{Teste}
De forma a testar a primeira regra, foi executada a aplicação $NoTintol$ em várias máquinas da rede, da seguinte forma:
\begin{lstlisting}[language=Java]
java NoTintol 10.101.204.4 1023 2000
\end{lstlisting}
Isto é, são criadas 2000 $threads$ por máquina atacante, que se tentam ligar simultaneamente ao porto 1023 do servidor $TintolmarketServer$. Verificamos que é gerado um alarme a cada intervalo de 60 segundos, independentemente do número de máquinas atacantes.\\
\\
Relativamente á segunda regra, o método de teste foi semelhante, sendo que foi executada a aplicação $NoTintol$ através do seguinte comando:
\begin{lstlisting}[language=Java]
java NoTintol 10.101.204.4 12345 2000
\end{lstlisting}
Sendo feita a uma alteração para que tenha como alvo o porto default do $TintolmarketServer$. Verificamos que ao executar este comando em cada máquina atacante é gerado um alerta por cada 15 segundos do ataque, por cada máquina atacante. Testamos também a execução do $Tintolmarket$ e verificamos que ao iniciar e terminar rapidamente o programa é gerado um alerta na quarta comunicação com o porto 12345.\\
\\
Durante os testes efetuados verificou-se que a ação da aplicação $NoTintol$ teve bastante impacto no servidor $TintolmarketServer$, visto que muito rapidamente se liga e desliga dos seus serviços. Isto resulta num consumo de recursos muito grande e consequente perda de performance do mesmo, podendo resultar numa $denial$ $of$ $service$ para clientes reais
\end{document}